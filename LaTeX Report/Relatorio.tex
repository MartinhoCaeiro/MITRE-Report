\documentclass[a4paper]{article}
% Pacotes necessários
\usepackage[portuguese]{babel}
\usepackage[backend=biber, style=apa, citestyle=apa, language=portuguese]{biblatex}
\usepackage{csquotes}
\addbibresource{Recursos/referencias.bib}

\usepackage{amsmath}
\usepackage{graphicx}
\usepackage{subcaption}
\usepackage{setspace}
\usepackage{siunitx} % Required for alignment
\sisetup{
  round-mode          = places, % Rounds numbers
  round-precision     = 2, % to 2 places
}
\usepackage{enumerate}
\usepackage{enumitem}
\usepackage{amsmath}
\usepackage{karnaugh-map}
\usepackage[section]{placeins}
\usepackage{geometry}
\usepackage{amssymb}
\usepackage{titling}
\usepackage[T1]{fontenc}
\usepackage{float}
\usepackage[hidelinks]{hyperref}
\usepackage{xcolor}
\usepackage{indentfirst}
\usepackage{array}
\usepackage{soul}
\usepackage{afterpage}
\newcolumntype{P}[1]{>{\centering\arraybackslash}p{#1}}
\onehalfspacing


% Comando para criar uma página vazia
\newcommand\myemptypage{
    \null
    \thispagestyle{empty}
    \addtocounter{page}{-1}
    \newpage
}

% Página de título principal
\newcommand{\firsttitlepage}{
    \begin{titlepage}
        \centering
        
        % Logos superior
        \begin{figure}[h!]
            \centering
            \includegraphics[width=6cm]{Recursos/Logos/LOGO_IPB.png} % Substitua pelo caminho da imagem
            \vspace{0.5cm}
        \end{figure}

        % Informações da instituição
        \large\textbf{INSTITUTO POLITÉCNICO DE BEJA} \\
        \large\textbf{Escola Superior de Tecnologia e Gestão} \\
        \large\textbf{Mestrado em Engenharia de Segurança Informática} \\
        \large\textbf{Fundamentos de Cibersegurança} \\
        
        \vspace{2cm}
        
        % Título do projeto
        {\Huge \textbf{Trabalho Individual}} \\
        
        \vspace{1.5cm}
        
        % Autores
        \large Martinho José Novo Caeiro - 23917 \\
        
        \vfill
        
        % Logo inferior
        \begin{figure}[h!]
            \centering
            \includegraphics[width=6cm]{Recursos/Logos/IPBejaESTIG.jpg} % Substitua pelo caminho da imagem
        \end{figure}
        
        \vspace{1cm}
        
        % Local e data
        {\large Beja, novembro de 2025}
    \end{titlepage}
}

\newcommand{\secondtitlepage}{
    \begin{titlepage}
        \centering
        \vspace*{1cm}
        
        % Informações da instituição
        \large\textbf{INSTITUTO POLITÉCNICO DE BEJA} \\
        \large\textbf{Escola Superior de Tecnologia e Gestão} \\
        \large\textbf{Mestrado em Engenharia de Segurança Informática} \\
        \large\textbf{Fundamentos de Cibersegurança} \\
        
        \vspace{2cm}
        
        % Título do projeto
        {\Huge \textbf{Trabalho Individual}} \\
        
        \vspace{1.5cm}
        
        % Autores
        \large Martinho José Novo Caeiro - 23917 \\

        \vspace{2cm}

        % Orientador
        \large Orientadores: Rui Miguel Silva \& Rogério Matos Bravo \\
        
        \vfill
        
        % Local e data
        {\large Beja, novembro de 2025}
    \end{titlepage}
}

\begin{document}


\pagenumbering{gobble} % Oculta numeração da página

% Primeira página de título
\firsttitlepage

\secondtitlepage


% Abstract
\section*{\LARGE\textbf{\textit{Resumo}}}

Este relatório descreve o desenvolvimento de uma aplicação de segurança informática, implementada em Python,
uma linguagem de programação dinâmica. O objetivo é explorar conceitos fundamentais de segurança
e demonstrar a aplicação prática desses conceitos. Esta aplicação é desenvolvida no âmbito da unidade curricular
de Fundamentos de Cibersegurança \cite{pagfc}.

\vspace{1cm}
% Keywords
\textbf{Keywords:} mitre, att\&ck, segurança informática, fundamentos de cibersegurança
\newpage
%--------------------------------------------------------------------------------------------------------------------------------------

\section*{\LARGE\textbf{\textit{Abstract}}}

This report describes the development of a cybersecurity application implemented in Python,
a dynamic programming language. The goal is to explore fundamental security concepts and demonstrate their practical application.
This application is developed within the scope of the Fundamentals of Cybersecurity course \cite{pagfc}.

\vspace{1cm}
% Keywords
\textbf{Keywords:} mitre, att\&ck, cybersecurity, fundamentals of cybersecurity


\renewcommand{\contentsname}{Índice}       % Título do sumário
\renewcommand{\listfigurename}{Índice de Figuras} % Título da lista de figuras

% Início do conteúdo do relatório
\newpage
\doublespacing
\tableofcontents
\listoffigures
\doublespacing

\newpage
\pagenumbering{arabic}

\section{Introdução}\label{intro}
Este relatório apresenta o trabalho individual realizado para a unidade curricular de
\textit{Fundamentos de Cibersegurança} do \textit{Mestrado em Engenharia de Segurança Informática} do
\textit{Instituto Politécnico de Beja}. O objetivo principal é a análise de conceitos e normas
(\textit{MITRE/ATT\&CK}, \textit{ISO 27000} e legislação nacional relevante).

Em termos de estrutura, o relatório organiza-se em dois blocos complementares. O \textbf{Capítulo I (Grupo I)} explora em detalhe a área seleccionada do MITRE/ATT\&CK, incluindo análise de campanhas representativas, identificação de táticas, técnicas e procedimentos (TTPs) e comparação de técnicas entre as matrizes Enterprise, Mobile e ICS. O \textbf{Capítulo II (Grupo II)} analisa os fundamentos da segurança da informação — pilares, vetores e o conceito de \textbf{Governança} — estabelecendo ligações práticas às normas \textit{ISO/IEC} e à legislação nacional quando pertinente, e propondo recomendações operacionais para deteção, mitigação, resposta a incidentes e gestão de risco.

A bibliografia e as fontes consultadas são apresentadas no final do documento e
o relatório será disponibilizado no repositório GitHub (\cite{github}).\\

%---------------------------------------------------------------------------------------------------------------------------
\section{Teoria}\label{theory}
\subsection{MITRE}
O MITRE (\cite{mitre}) é uma organização sem fins lucrativos que opera centros de pesquisa e desenvolvimento financiados pelo governo dos Estados Unidos.

\subsection{Att\&ck}
O ATT\&CK (\cite{attack}) é um framework desenvolvido pelo MITRE que documenta as táticas e técnicas utilizadas por adversários cibernéticos.

\subsection{Segurança da Informação}
A Segurança da Informação (\cite{secinfo}) refere-se à prática de proteger informações e sistemas de informação contra acesso não autorizado, uso, divulgação, interrupção, modificação ou destruição.
Envolve a implementação de políticas, procedimentos e tecnologias para garantir a confidencialidade, integridade e disponibilidade dos dados.

%---------------------------------------------------------------------------------------------------------------------------
\newpage
\section{Grupo I}
O presente capítulo procede à análise detalhada das áreas selecionadas do MITRE/ATT\&CK, com especial foco na transformação da informação técnica em inteligência acionável para operações de defesa. Pretende-se identificar e descrever as táticas, técnicas e procedimentos (TTPs) mais relevantes, analisar campanhas representativas e comparar técnicas entre as diferentes matrizes (Enterprise, Mobile e ICS). O capítulo integra estudos de caso, avaliação de ferramentas e recomendações práticas para deteção, mitigação e resposta a incidentes, promovendo a ligação entre a análise de ameaças e as medidas operacionais.
%---------------------------------------------------------------------------------------------------------------------------
\subsection{Cyber Intelligence Threat Analysis}

A área \textit{Cyber Intelligence Threat Analysis} (\cite{cita}) centra-se na recolha, organização e análise estruturada de informação sobre ameaças cibernéticas, com o objetivo de produzir \textbf{inteligência acionável} que apoie a defesa, a deteção e a resposta a incidentes. Esta análise procura compreender os adversários, as suas motivações, capacidades e métodos de ataque, permitindo antecipar comportamentos e melhorar o nível de proteção das organizações.

A análise de ameaças baseia-se na identificação de grupos e atores maliciosos, no estudo das suas \textbf{táticas, técnicas e procedimentos (TTPs)}, e na correlação de indicadores de compromisso e outros artefactos de ataque. Para isso, a área utiliza várias estruturas desenvolvidas pelo MITRE, entre as quais se destacam:

\begin{itemize}
	\item \textbf{STIX} - representação estruturada de ameaças e observáveis cibernéticos.
	\item \textbf{CVE} - identificação de vulnerabilidades conhecidas.
	\item \textbf{CPE} - descrição de plataformas e sistemas afetados.
	\item \textbf{CWE} - categorização de fraquezas de software.
	\item \textbf{MAEC} - caracterização de malware.
	\item \textbf{CAPEC} - descrição de padrões de ataque observados.
\end{itemize}

A \textit{Cyber Intelligence Threat Analysis} é fundamental para transformar dados técnicos dispersos em conhecimento útil, permitindo reforçar mecanismos de deteção, apoiar decisões de segurança e orientar a resposta a incidentes com base em evidência real. Esta área articula-se naturalmente com domínios como a partilha de informação de ameaças e a coordenação de incidentes, contribuindo para uma postura de segurança mais preventiva e informada.

%---------------------------------------------------------------------------------------------------------------------------
\newpage
\subsection{Cyber Threat Information Sharing}

A área \textit{Cyber Threat Information Sharing} (\cite{ctis}) foca-se nos processos e mecanismos utilizados para partilhar informação sobre ameaças cibernéticas entre equipas, organizações e entidades externas. O objetivo principal é aumentar a capacidade coletiva de deteção, prevenção e resposta, garantindo que os indicadores de ataque, técnicas utilizadas pelos adversários e outros elementos relevantes chegam rapidamente aos intervenientes que deles necessitam.

A partilha de informação de ameaças pode incluir \textbf{IOCs} (Indicators of Compromise), padrões de comportamento observados, artefactos recolhidos em incidentes, ou relatórios de análise produzidos por equipas de \textit{Cyber Threat Intelligence}. Esta partilha deve seguir formatos estruturados e normalizados que permitam interoperabilidade e automatização.

Entre os sistemas e padrões mais utilizados destacam-se:

\begin{itemize}
	\item \textbf{TAXII} - framework comunitário que define conceitos, protocolos e trocas de mensagens para partilha segura e automatizada de informação de ameaças.
	\item \textbf{STIX} - linguagem para representar informação de ameaças de forma estruturada e interoperável.
\end{itemize}

A área \textit{Cyber Threat Information Sharing} complementa diretamente a \textit{Cyber Intelligence Threat Analysis}, uma vez que a inteligência produzida pela análise de ameaças só alcança o seu valor máximo quando é partilhada de forma eficiente e segura. Esta cooperação melhora a defesa global, reduz o tempo de resposta e promove uma postura mais colaborativa no combate ao cibercrime.

%---------------------------------------------------------------------------------------------------------------------------
\newpage
\subsection{Campanha \textit{Operation MidnightEclipse}}

A campanha \textbf{Operation MidnightEclipse - ID C0048} (\cite{midnighteclipse}) decorreu entre março e abril de 2024.
Esta explorou a vulnerabilidade CVE-2024-3400 no módulo GlobalProtect de firewalls Palo Alto, permitindo execução remota de código com privilégios de root.
O ator associado, identificado como UTA0218, realizou um ataque direcionado, combinando exploração de zero-day, implantação de backdoor e exfiltração de dados críticos.

As técnicas e TTPs observadas incluem:

\begin{itemize}
	\item T1190 - Exploit Public-Facing Application: exploração da vulnerabilidade no firewall.
	\item T1059.004 - Command and Scripting Interpreter: Unix Shell: execução de comandos via bash.
	\item T1105 - Ingress Tool Transfer: download de ferramentas adicionais.
	\item T1053.003 - Scheduled Task/Job: Cron: persistência com tarefas agendadas.
	\item T1078.002 - Valid Accounts: Domain Accounts: uso de credenciais válidas para movimentação lateral.
	\item T1090 - Proxy: comunicação C2 via túnel com GOST.
\end{itemize}

O software/ferramentas usadas incluiu:
\begin{itemize}
	\item UPSTYLE: backdoor Python para execução remota.
	\item GOST: criação de túneis reversos para comunicação com servidores de comando e controlo.
\end{itemize}

A campanha visou roubo de credenciais, dados de configuração do firewall e movimentação lateral dentro da rede, mantendo persistência com cron jobs e backdoors. A mitigação recomendada incluiu patches do PAN-OS, monitorização de tráfego, análise de logs e recolha de evidências forenses.

%---------------------------------------------------------------------------------------------------------------------------
\newpage
\subsection{Tática Collection}

A tática \textit{Collection} descreve a recolha de dados realizada por um adversário após comprometer um sistema. Embora a tática seja comum às matrizes \textit{Enterprise}, \textit{Mobile} e \textit{ICS}, as técnicas associadas diferem significativamente devido às particularidades de cada ambiente.

\subsubsection{Enterprise}
No ambiente \textit{Enterprise - TA0009} (\cite{colent}), a recolha de dados centra-se em artefactos típicos de sistemas operativos tradicionais, como ficheiros locais, credenciais, histórico de navegação, processos, memória e dados de aplicações corporativas. As técnicas focam mecanismos amplamente presentes em sistemas Windows, Linux e macOS, refletindo um conjunto de ativos essencialmente digitais.

\subsubsection{Mobile}
Na matriz \textit{Mobile - TA0035} (\cite{colmob}), a recolha é condicionada por mecanismos de segurança próprios de dispositivos móveis, tais como \textit{sandboxing}, permissões de aplicações e acesso limitado ao sistema de ficheiros. Assim, as técnicas incidem sobre dados pessoais e de aplicações, nomeadamente localização, contactos, mensagens, fotos, sensores e dados armazenados por aplicações móveis.

\subsubsection{ICS}
No domínio \textit{ICS - TA0100} (\cite{colics}), a recolha de informação está relacionada com processos industriais e sistemas de controlo. As técnicas visam obter dados operacionais provenientes de PLCs, sensores, actuadores, sistemas SCADA e parâmetros de controlo. Estes dados são críticos para compreender e manipular processos físicos.

\subsubsection{Exemplos}
A título ilustrativo:
\begin{itemize}
	\item \textbf{Enterprise:} a técnica \textit{T1005 - Data from Local System} permite recolher ficheiros e credenciais armazenadas num sistema.
	\item \textbf{Mobile:} a técnica \textit{T1409 - Access Stored Application Data} foca-se na recolha de dados de aplicações móveis, como mensagens ou bases de dados internas.
	\item \textbf{ICS:} a técnica \textit{T0887 - Operation Information} permite ao adversário recolher dados operacionais de dispositivos industriais, como leituras de sensores ou estados de PLCs.
\end{itemize}

Estas diferenças demonstram que, embora a tática seja comum, as técnicas variam devido à natureza distinta dos ambientes e dos dados relevantes em cada matriz.


%---------------------------------------------------------------------------------------------------------------------------
\newpage
\section{Grupo II}
O presente capítulo tem como objetivo abordar os principais conceitos estruturantes da \textbf{Segurança da Informação}, tal como definidos no âmbito da unidade curricular de \textit{Fundamentos de Cibersegurança}.
São analisados os pilares, vetores e princípios de governança que sustentam a proteção da informação e a gestão de riscos no contexto das organizações modernas, com particular atenção às normas da família \textit{ISO/IEC 27000} e à legislação nacional aplicável.

%---------------------------------------------------------------------------------------------------------------------------

\subsection{Os Quatro Pilares da Segurança da Informação}
A segurança da informação, de acordo com uma visão abrangente e integrada, assenta em quatro pilares fundamentais: \textbf{Tecnologias}, \textbf{Pessoas}, \textbf{Organizações} e \textbf{Segurança Física}. Estes elementos, interdependentes entre si, formam a base sobre a qual as normas da família \textit{ISO/IEC 27000} (incluindo as versões mais recentes da \textit{ISO/IEC 27001:2022} e \textit{ISO/IEC 27002:2022}) estruturam a gestão da segurança da informação.

\subsubsection{Tecnologias}
Este pilar corresponde ao conjunto de ferramentas, sistemas e mecanismos técnicos implementados para proteger a informação. Inclui medidas como o controlo de acessos, a encriptação, a gestão de vulnerabilidades, os sistemas de deteção e prevenção de intrusões, bem como políticas de \textit{backup} e recuperação.
O foco é garantir que os recursos tecnológicos oferecem \textbf{confidencialidade}, \textbf{integridade} e \textbf{disponibilidade}, de acordo com os objetivos organizacionais e as boas práticas definidas pela \textit{ISO/IEC 27002:2022}.

\subsubsection{Pessoas}
As pessoas representam simultaneamente o \textbf{maior ativo} e o \textbf{elo mais vulnerável} da segurança da informação. A consciencialização, a formação contínua e a definição clara de responsabilidades são essenciais para reduzir o risco humano.
De acordo com as normas ISO, a cultura organizacional deve promover comportamentos seguros e uma compreensão clara das políticas internas de segurança, prevenindo negligência, erro humano ou engenharia social.

\subsubsection{Organizações (processos e procedimentos)}
Este pilar abrange a estrutura organizacional, os processos e os procedimentos formais que sustentam o \textbf{Sistema de Gestão da Segurança da Informação (SGSI)}.
Inclui políticas, planos de gestão de incidentes, auditorias, avaliação de riscos e conformidade com a legislação (como o RGPD e a legislação nacional aplicável).
A norma \textit{ISO/IEC 27001:2022} reforça este pilar ao definir requisitos para a implementação e manutenção de controlos de segurança eficazes, sustentados em documentação e melhoria contínua.

\subsubsection{Segurança Física}
A segurança física visa proteger as infraestruturas, equipamentos e suportes de informação contra ameaças físicas - como acesso não autorizado, incêndios, inundações ou sabotagem.
Abrange o controlo de acessos a edifícios, a vigilância, a gestão ambiental e a proteção dos dispositivos de armazenamento.
Sem segurança física, qualquer sistema técnico ou processo organizacional fica vulnerável, comprometendo os restantes pilares.

\subsubsection{Pilar mais importante}
Apesar da sua interdependência, o \textbf{pilar das pessoas} é frequentemente considerado o mais determinante.
As tecnologias, políticas e infraestruturas só são eficazes se forem corretamente compreendidas e aplicadas pelos utilizadores. O comportamento humano é o fator crítico que pode tanto reforçar como comprometer os restantes pilares, tornando a formação e a sensibilização indispensáveis à eficácia global da segurança da informação.

\subsubsection{Ligação à Intervenção Digital Forense}
A \textbf{intervenção digital forense} - responsável pela recolha, preservação e análise de evidências digitais - relaciona-se diretamente com vários destes pilares, mas de forma especial com as \textbf{tecnologias} e as \textbf{organizações (processos e procedimentos)}.

\begin{itemize}
	\item \textbf{Tecnologias:} a recolha e preservação de evidências requerem ferramentas técnicas adequadas, como software de aquisição forense e mecanismos de hashing, que asseguram a integridade dos dados.
	\item \textbf{Organizações:} a existência de procedimentos normalizados (cadeia de custódia, registos de auditoria, políticas de acesso e conservação) garante que a prova digital é admissível e fidedigna.
	\item \textbf{Pessoas:} os peritos forenses e os técnicos de segurança devem agir de forma ética e tecnicamente rigorosa, assegurando a imparcialidade e a rastreabilidade das suas ações.
\end{itemize}

Assim, a intervenção digital forense concretiza a aplicação prática dos pilares da segurança da informação, garantindo que a gestão de incidentes e a produção de prova digital são realizadas de forma segura, controlada e conforme às normas internacionais.

%---------------------------------------------------------------------------------------------------------------------------
\newpage
\subsection{Os Três Vetores da Segurança da Informação}
A segurança da informação pode ser analisada segundo três vetores principais, também designados como as \textbf{três dimensões operacionais da segurança}: \textbf{Segurança Física}, \textbf{Segurança Humana} e \textbf{Segurança Lógica}.
Estes vetores formam uma estrutura integrada que assegura a proteção da informação em todas as suas formas - material, humana e tecnológica - e encontram correspondência direta no enquadramento da \textbf{Segurança da Informação Classificada (SIC)}, conforme definido pela \textbf{Resolução do Conselho de Ministros n.º 5/1990}, que aprova a \textbf{SEGNAC4} (Sistema de Segurança Nacional de Classificação, Codificação e Salvaguarda de Informação Classificada).

\subsubsection{Segurança Física}
A segurança física tem como objetivo proteger as instalações, equipamentos e suportes de informação contra ameaças de natureza física, acidental ou intencional.
Inclui medidas como o controlo de acessos a edifícios e zonas restritas, vigilância eletrónica, barreiras físicas, proteção ambiental (contra incêndios, inundações, etc.) e a salvaguarda de documentos em cofres ou armários classificados.
No contexto da \textbf{SEGNAC4}, a segurança física é indispensável para garantir que a informação classificada, em suporte material, não é acedida, copiada ou destruída sem autorização.

\subsubsection{Segurança Humana}
A segurança humana refere-se à gestão dos riscos associados ao fator humano, reconhecendo que as pessoas podem ser tanto a maior defesa como a maior vulnerabilidade da segurança da informação.
Abrange procedimentos de seleção, credenciação e formação de pessoal, garantindo que apenas indivíduos devidamente autorizados e conscientes das suas responsabilidades têm acesso a informação classificada.
A \textbf{SEGNAC4} estabelece regras específicas sobre credenciação de segurança, dever de sigilo e responsabilidade disciplinar ou penal em caso de violação das normas de proteção da informação classificada.

\subsubsection{Segurança Lógica}
A segurança lógica, também designada \textbf{segurança tecnológica ou digital}, incide sobre os sistemas informáticos e redes de comunicação.
Compreende o conjunto de medidas destinadas a proteger a informação processada ou armazenada em formato eletrónico, incluindo autenticação, controlo de acessos, encriptação, gestão de vulnerabilidades, auditorias de segurança e registos de atividade.
Na \textbf{SIC}, a segurança lógica é essencial para assegurar que a informação classificada mantida em sistemas digitais cumpre os níveis de proteção definidos, prevenindo o acesso não autorizado ou a exfiltração de dados.

\subsubsection{Ligação à SEGNAC4}
A \textbf{Resolução do Conselho de Ministros n.º 5/1990} define o modelo nacional de proteção da informação classificada, estabelecendo que a segurança deve ser assegurada de forma global, integrando os três vetores mencionados.
A eficácia da \textbf{Segurança da Informação Classificada} depende, assim, da articulação entre as dimensões física, humana e lógica - cada uma cobrindo diferentes fases e contextos da proteção da informação.
Quando devidamente coordenadas, estas três dimensões garantem a \textbf{confidencialidade}, \textbf{integridade} e \textbf{disponibilidade} da informação classificada, em conformidade com as exigências nacionais e internacionais de segurança.


%---------------------------------------------------------------------------------------------------------------------------
\newpage
\subsection{O conceito de ‘Governança’}

O conceito de \textbf{Governança} da Segurança da Informação e da Cibersegurança, conforme apresentado no âmbito deste curso e nas normas da família \textit{ISO/IEC 27001}, representa o conjunto de práticas, responsabilidades e processos que asseguram que a gestão da segurança é conduzida de forma estruturada, mensurável e alinhada com os objetivos estratégicos da organização.

Mais especificamente, os controlos \textbf{8.15} e \textbf{8.16} da \textit{ISO/IEC 27001:2022} estabelecem as bases da governança, determinando que as organizações devem \textbf{monitorizar, rever e melhorar continuamente} os mecanismos de segurança, assegurando que as medidas implementadas permanecem eficazes e adequadas ao contexto operacional e às ameaças em evolução.

\subsubsection{Conteúdo e Aplicação Prática}

A \textbf{Governança da Segurança da Informação} implica:
\begin{itemize}
	\item \textbf{Definir a granularidade dos acessos e dos “assets críticos”}, determinando níveis de privilégio e identificando os recursos cuja proteção é prioritária;
	\item \textbf{Avaliar e testar} regularmente a eficácia dos controlos e políticas de segurança;
	\item \textbf{Monitorizar} o comportamento dos sistemas e dos utilizadores, através de mecanismos de auditoria, registos e indicadores de desempenho;
	\item \textbf{Testar e rever os SOP (Standard Operating Procedures)}, garantindo que os procedimentos operacionais estão atualizados, coerentes e eficazes na mitigação de riscos.
\end{itemize}

Estas atividades asseguram que a segurança da informação é gerida de forma sistemática e não meramente reativa, promovendo uma cultura de responsabilidade, conformidade e melhoria contínua.

\subsubsection{Importância para a Cibersegurança e o Combate ao Cibercrime}
No domínio da \textbf{Cibersegurança}, a Governança é essencial para transformar políticas e orientações estratégicas em práticas concretas e auditáveis.
Permite estabelecer mecanismos de responsabilização, definir papéis claros (por exemplo, CISO, gestores de risco, auditores) e integrar a gestão da segurança com os objetivos institucionais.

No \textbf{combate ao cibercrime}, a governança contribui para a capacidade de resposta organizada a incidentes, garantindo rastreabilidade, preservação de evidências digitais e cumprimento de obrigações legais - aspetos fundamentais em processos de investigação e cooperação entre entidades públicas e privadas.

\subsubsection{Correspondência com a RCM n.º 41/2018}
A \textbf{Resolução do Conselho de Ministros n.º 41/2018}, que aprova a \textbf{Estratégia Nacional de Segurança do Ciberespaço (ENSCE)}, apresenta um enquadramento convergente com o conceito de governança definido nas normas ISO.
Ambos os documentos enfatizam:
\begin{itemize}
	\item A necessidade de \textbf{estruturas organizacionais de coordenação} e de \textbf{responsabilidade partilhada};
	\item A importância da \textbf{monitorização e avaliação contínua} das políticas de segurança;
	\item A criação de \textbf{mecanismos de supervisão e reporte} de incidentes e vulnerabilidades;
	\item A promoção de uma \textbf{cultura de segurança} transversal ao setor público, privado e académico.
\end{itemize}

Assim, pode afirmar-se que existe uma \textbf{correspondência direta} entre o conceito de Governança, tal como definido nas normas \textit{ISO/IEC 27001:2022 (8.15/8.16)}, e os princípios orientadores da \textbf{RCM n.º 41/2018}, sendo ambos instrumentos complementares na consolidação da cibersegurança e na prevenção e combate ao cibercrime em Portugal.


%---------------------------------------------------------------------------------------------------------------------------
\newpage
\section{Conclusão}\label{con}

%---------------------------------------------------------------------------------------------------------------------------

\newpage
\renewcommand{\refname}{Bibliografia} % Para artigos
\renewcommand{\bibname}{Bibliografia} % Para livros e relatórios
\addcontentsline{toc}{section}{Bibliografia} % Adiciona a Bibliografia ao índice
\printbibliography

\end{document}
